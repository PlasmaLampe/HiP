\chapter[Introduction into the thesis]{Introduction in the thesis}
This first chapter will introduce the benefits of the system that we will develope within this thesis and will explain the current situation without the system.

\section{What is the current situation without the tool/app}
At the present point in time, guests of the city Paderborn has to look up information about the city in a tedious way, for example using Wikipedia or other existing platforms. On the other hand, people, who want to provide information about the city (e.g., university employees or students), have to provide these information in a general accessible way, again like Wikipedia. Thus, they are limited to the features that are provided by these platforms. Since Wikipedia has been founded in 2004 by the Wikimedia foundation (\cite{wikimedia}), most of the used technologies of the web application are nowadays outdated and very general. So, there is a rising need for a new technological updated approach, which is more focused on the specific topic of the city Paderborn and its history. 

Especially the use of mobile devices has been risen in this time, which is easily recognizable at the number of sales of the Apple iPhone. The iPhone has been sold 0.27 million times in Q3 2007 and 51.03 million times in Q1 2014 (\cite{statIPhone}). Of course, this shift from the device side (i.e., hardware) includes a major shift in (software-) technology as well. Technologies like the nowadays well known \ac{AR} would not be possible in 2004. Of course, this new technology includes a lot of new opportunities to transfer knowledge between people and cultures. \ac{AR} is a great example to show how \emph{the real world} is blended more and more with artificial information; for example in form of call-outs and layers. \ac{AR} is a technology that displays virtual (i.e., digital) information on top of a real object or location using the camera of a mobile device as input for the real objects. So, it ends up to be a combination of both worlds; the real and the digital one. Azuma et. al. has shown a lot of possible fields where the usage of \ac{AR} would be a great improvement, which includes the field of annotation and visualization (\cite{Azu97}). Furthermore, path finding and navigation are fields that could be revolutionized by using \ac{AR} on mobile devices.

With a simple information providing website or app like Wikipedia, we include the tedious situation that the person that wants to get to the place he just read about needs to input the address into another app to navigate him to the position. After he have arrived, he need to switch back to, for example, Wikipedia to manual compare the written information with the object or place he sees in front of him. If the person wants to change the shown information on his mobile device, he does this in general by using the touchscreen of the device. Nevertheless, he is comparing and looking at something that is placed in front of him. This leads to a break within the action and perception space (\cite{ham01}) and is a bad example with respect to the locality of the information (\cite{Bon10}). As we will see, \ac{AR} is a tool that we can use to remove this problem and join the action and perception space while keeping the locality of the information in mind.  Furthermore, at the moment we have a lot of unnecessary overhead due to the needed app switching between the information app and the navigation app.   

Now, even if somebody wants to publish information about Paderborn on Wikipedia to enable guests of the city to get knowledge about the environment, it is only possible to publish the information as static content (that includes text, graphics, audio and video). On top of that, it is not possible to review the information privately and in enough detail to create university courses that do not include a written paper as the final exam but an entry within such an information system.  So, if we would have private annotations within a system that is owned by the university, it would enable the university employees to offer courses that fill the database about Paderborn with high quality content by students.

This leads us to the application that should be prototypical developed within this master thesis, which will be described in the next section. 

\section{What would the system look like (briefly)}
As we will see in chapter \ref{draft}, the system will be divided in two big parts. One part is the web-backend, which is connected to an \textit{MongoDB} to store and retrieve the needed information. This backend will provide a \ac{REST} interface, which enables it to be connected two different kind of frontends. 

These frontends create the second big part of the system. The first prototype of the system will include a web-frontend to access the backend for administrative purposes (e.g., including new data by students, creating groups, review data, etc.) and will be driven by the play framework in combination with AngularJS. 

The second kind of frontend, which will be sketched within this thesis, will be the smartphone frontend. With this frontend, the end-users (i.e., everybody who downloads the app from the app-store) are able access the information, which are included in the backend. Furthermore, the smartphone frontend will make use of \ac{AR} features to show the information that is stored in the backend. 

After we have now seen, how the system will look like, we will now take a short look at the question, who will actually benefit of such a system.   

\section{What would be better if the app would exit? Who would benefit?}
On the one hand, users would benefit from the app by having a neat tool to discover the history of the city paderborn. It will be a great experience to be guided trough the city and learn a lot of important and interesting facts about the environment. Furthermore, we would prevent the problem that people are sitting next to an object (e.g., within a museum) but are using the app without getting in touch with the real-world object because the \ac{AR} functionality induces activity of the user.  

On the other hand, the system will be a nice variation for the students, which may be bored from the typical \textit{send in a homework to pass the exam} cycle and can include the information directly into the backend and are able to see \textit{their} work some time later via the app in the frontend. So, they are actually able to \textbf{do} something, which is used in the future.    

\section{Outline}
This master thesis contains six chapters. The first chapter contained an introduction and ends with this outline.

The second chapter will explain all needed fundamentals of this Master thesis in detail and will show the used frameworks and tools.

The third chapter will outline the application design and describe some general design decisions.

The fourth chapter will show important parts of the actual implementation, used tools and the final \acf{UML} diagrams of the application.

The fifth chapter will show unit tests and the results of a survey that was used to evaluate usability of the system. 

The sixth and last chapter will deal with arisen problems and will discuss the development for future work.
